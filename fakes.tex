\section{Same sign non-prompt background}
Here we consider the case of the non-prompt leptons that come from heavy
flavour decays, decays in flight, or conversions.

Since we are requiring two leptons in the final state we will have
$N_{pp}$ events with both prompt leptons, $N_{fp}$ with one prompt and one
fake lepton and $N_{ff}$ where both leptons are fake.

In general the largest source of the $tf$ contribution is semi-leptonic top
or $\mathrm{W}$ and $\mathrm{Z}$ plus jets events where the second lepton comes from a heavy flavour decay.

The $ff$ contribtution is dominated by multijet QCD events. The data driven calculation of the fake contribution
is done with the tight-loose method described in Ref. [8].

The procedure is in two steps: first
the average fake rate is determined using control samples enriched in non-prompt leptons.
Secondly the fake rate is applied to the events selected by the analysis.
The fake rate $f$ is defined
as the ``fake ratio'', that is the probability for a loose lepton to also pass the tight identification selection.

\begin{equation*}
    f = \dfrac{#tight}{#loose}
\end{equation*}

The following assumptions are made:
\begin{itemize}
    \item the probability that a lepton is fake is independent from the
        presence of another lepton;
    \item the probability that a prompt lepton passes both the loose and the
        tight selections is 1;
    \item the fake rate as measured in an inclusive QCD sample as described
        in Ref.[8] is applicable to the case of a QCD object in a lepton
        plus jet event;
    \item no correlation is included for the fake-fake lepton case;
\end{itemize}

The leptons and jets are defined as follows:
\begin{description}
    \item[electron]
        \begin{description}
            \item[loose]
                \begin{itemize}
                    \item $p_t > \unit[30]{GeV}$;
                    \item $\lvert \eta \rvert < 2.1$;
                    \item not in barrel-endcap gap;
                    \item charge consistent;
                    \item cut category: SuperTightID;
                    \item relative isolation $< 0.6$.
                \end{itemize}
            \item[tight] 
                \begin{itemize}
                    \item relative isolation $< 0.2$;
                    \item conversion $\Delta \cot \theta > 0.02$ or distance $> 0.02$.
                \end{itemize}
        \end{description}
    \item[muon] 
        \begin{description}
            \item[loose]
                \begin{itemize}
                    \item $p_t > \unit[30]{GeV}$;
                    \item $\lvert \eta \rvert < 2.1$;
                    \item at least one valid tracker hit;
                    \item normalized $\chi^2 < 50$;
                    \item impact parameter $< \unit[2]{cm}$;
                    \item relative isolation $< 0.4$;
                \end{itemize}
            \item[tight] 
                \begin{itemize}
                    \item at least ten valid tracker hits;
                    \item normalized $\chi^2 < 10$;
                    \item at least one matching segment;
                    \item impact parameter $< \unit[0.02]{cm}$;
                    \item relative isolation $< 0.2$;
                \end{itemize}
        \end{description}
    \item[jet] 
        \begin{itemize}
            \item separated from any lepton by $\Delta R > 0.3$;
            \item $p_t > \unit[10]{GeV}$;
            \item $\lvert \eta \rvert < 2.4$;
        \end{itemize}
\end{description}

The data samples used to determine the fake rate are selected with the following cuts:
\begin{itemize}
    \item scraping, good vertex;
    \item at least one jet;
    \item at least one loose lepton;
    \item $E_t^{\text{miss}} < \unit[20]{Gev}$;
    \item $m_t < \unit[25]{Gev}$;
    \item $\mathrm{Z}$ veto: reject events if $71 < m_{\ell \ell} < 111$.
\end{itemize}

In order to predict the final contribution to the background of
$N_{tf} + N_{ff}$ we use the technique described in
Ref. [9] and [8] to relate them to the measurable numbers $N_{ti}$
($i = 0,1,2$) with 0,1 or 2 leptons passing the tight cuts, the remaining
ones failing these cuts.

The contributions are:
\begin{align*}
    \varepsilon &= \frac{f}{1 - f}\\
N_{tf} &= \varepsilon N_{t1} - 2 \varepsilon^2 N_{t0}\\
N_{ff} &= \varepsilon^2 N_{t0}
\end{align*}
    
\subsection{Closure tests}
