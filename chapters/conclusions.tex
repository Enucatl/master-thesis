\chapter*{Conclusions}
The search for new physics at the LHC has entered a very challenging phase:
the large amount and quality of the collected data allow to search for many
signatures of physics beyond the Standard Model.

The techniques developed in this thesis for the analysis of signal events with the razor variables have been proven to be
effective. The kinematical properties of these variables are very general and can be
applied to a wide range of searches for new physics. It is the first
time the razor variables are shown to be an accurate estimate for the energy
scale of a process by considering only part of the decay chain, while
previous searches only included them as global, event-by-event quantities.

This is particularly
relevant for the analysis of the large amount of data the CMS experiment has
recorded in 2012, amounting to $\unit[15]{fb^{-1}}$. A more detailed event
description could possibly make the difference while looking for minute
deviations from Standard Model expectations.

The analysis presented in this work is a search for heavy partners of the
top quark in same-sign dilepton events. The data collected in 2011 by the
CMS experiment at a centre-of-mass energy of \unit[7]{TeV} are analysed. A
counting experiment is established by selecting events in three decay
channels: \E\E, \E\M\ and \M\M. The Standard Model backgrounds are estimated
with different Monte Carlo and data-driven techniques.

The observed yields agree with the Standard Model prediction. Hence, no
evidence for new physics is found, and lower limits on the mass of the top
partner are set.
