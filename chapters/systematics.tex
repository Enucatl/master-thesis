\chapter{Systematic uncertainties}
Systematic uncertainties need to be taken into account because of the
imperfect knowledge of either detector effects or theory.
Detector effects include uncertainties in the trigger efficiencies, the
efficiency of the lepton reconstruction, the jet energy correction factors
and pile-up conditions.

Lepton reconstruction efficiencies were studied in~\cite{CMS-EXO-12-001}. The uncertainty on the trigger
efficiency is taken to be 5\%. This is a conservative estimate based on the study of the dilepton triggers in~\cite{bprime2011}. We also include a 2.2\% uncertainty 
on the luminosity~\cite{CMS-SMP-12-008}. These systematic uncertainties are summarized in Table~\ref{tab:Systematics}.

\begin{table}
    \centering
\begin{tabular}{*2c}
    \toprule
          effect & uncertainty (\%) \\
          \midrule
  electron trigger          & 5.0 \\
  muon trigger              & 5.0 \\
  lepton efficiency         & 3.0 \\
  luminosity                & 2.2 \\
  \midrule
\end{tabular}
\caption{Systematic uncertainties.}
\label{tab:Systematics}
\end{table}

The jet energy scale (JES) and pile-up uncertainties are calculated by varying the JES and pileup according to the recommended recipes in the samples for which 
we take the result from Monte Carlo. The effect of these depends on the topology of the sample. The JES and pileup uncertainties for the backgrounds are 
summarized in Table~\ref{tab:MConlySystematics}. For the signal points, we
take the JES and pile-up uncertainties to be the maximum value of these for single
point. This corresponds to 0.9\% for the JES and 2\% for the pileup. Table~\ref{tab:MConlySystematics} also
contains the overall normalization uncertainty for each background sample.
The $\W\Z$ and $\Z\Z$ normalization uncertainties are taken from~\cite{CMS-AN-2011-333}. 
For the rare backgrounds, we assume a normalization uncertainty of 50\%, as
these processes have never been observed at the LHC or have not been
measured well. We also include a 50\% uncertainty associated with the
estimation of the non-prompt lepton background~\ref{sec:closure}
and a 20\% uncertainty for the charge misidentification. This comes from the
difference between the expected number of events derived with the method
described in~\ref{sec:charge_misid} and a more sophisticated $\eta$ dependent estimate. 

The inputs to the calculation of the limits on the $T_{5/3}$ mass affected by the systematic uncertainties are the luminosity, the uncertainty on the signal efficiency 
and the uncertainty on the total background. The latter is the largest of the three and it is dominated by the 50\% uncertainty associated with the estimation of the 
non-prompt lepton background and the normalization uncertainty on the rare backgrounds.

\begin{table}[htb]
    \centering
\begin{tabular}{*4c}
    \toprule
    sample       &    JES & pile-up & normalization \\
    \midrule
$\W\Z$             & 5.0\%  & 1.8\%  & 17\%    \\
$\Z\Z$             & 1.1\%  & 0.6\%  & 7.5\% \\
$\W^{\pm}\W^{\pm}$ & 4.5\%  & 2.4\%  & 50\% \\
$\W\W\W$            & 3.7\%  & 0.5\%  & 50\% \\
$\ttbar \W$       & 3.4\%  & 0.94\% & 50\% \\
$\ttbar \Z$       & 3.7\%  & 0.25\% & 50\% \\
\bottomrule
\end{tabular}
\caption{Systematic uncertainties for backgrounds that are taken from Monte Carlo.}
\label{tab:MConlySystematics}
\end{table}

